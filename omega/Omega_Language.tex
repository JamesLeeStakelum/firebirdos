\documentclass{article}
\usepackage[utf8]{inputenc}
\usepackage{geometry}
\geometry{a4paper, margin=1in}
\usepackage{amsmath}
\usepackage{amsfonts}
\usepackage{amssymb}
\usepackage{graphicx}
\usepackage{hyperref}
\usepackage{listings}
\usepackage{xcolor}
\usepackage{textcomp}

\hypersetup{
    colorlinks=true,
    linkcolor=blue,
    filecolor=magenta,      
    urlcolor=cyan,
}

% Define colors for code listings
\definecolor{codegreen}{rgb}{0,0.6,0}
\definecolor{codegray}{rgb}{0.5,0.5,0.5}
\definecolor{codepurple}{rgb}{0.58,0,0.82}
\definecolor{backcolour}{rgb}{0.95,0.95,0.92}

% --- Global settings for all code listings ---
\lstset{
    backgroundcolor=\color{backcolour},
    commentstyle=\color{codegreen},
    keywordstyle=\color{blue}\bfseries,
    numberstyle=\tiny\color{codegray},
    stringstyle=\color{codepurple},
    basicstyle=\ttfamily\footnotesize,
    tabsize=2,
    captionpos=b,
    keepspaces=true,
    numbers=left,
    numbersep=5pt,
    showspaces=false,
    showstringspaces=false,
    showtabs=false,
    frame=single,
    % --- KEY SETTINGS FOR LINE WRAPPING ---
    breaklines=true,
    breakatwhitespace=false,
    breakindent=1em,
    xleftmargin=2em,
    % Simplified postbreak command to avoid syntax issues
    postbreak=\mbox{\tiny\ensuremath{\hookrightarrow}}
}


% Define Omega language style
\lstdefinelanguage{Omega}{
    keywords={MODULE, IF, LOOP, FUNCTION, DATA_TYPE_SCHEMA, DEFINITION, VAR, SEMANTIC_PROPERTIES, PII_Data, RESOURCE_BOUND, SUBJECT, TYPE, METRIC, THRESHOLD, VIOLATION_POLICY, GOVERNANCE_RULE, SCOPE, PREDICATE, ENFORCEMENT_CONTEXT, PRIORITY, ActionIsDelete, TargetSchemaIs, UserProfile, ApiClient_Service, ApiCalls, CallsPerMinute, ThrottleAndLogPolicy, UserDataManagement, PreventActionAndAlertAdmin}
}

% Define EBNF language style
\lstdefinelanguage{EBNF}{
    keywords={MODULE, END, MODULE, BLOCK, STATEMENT, IF, THEN, ELSE, END, IF, LOOP, WHILE, FOR, IN, TO, END, LOOP, FUNCTION, RETURNS, END, FUNCTION, DECLARE, VAR, AS, ASSIGN, RETURN, BREAK, CONTINUE, BOOLEAN, INTEGER, FLOAT, STRING, TRUE, FALSE, VOID, CONTEXT_RULE, TEMPORAL_RELATION, RESOURCE_BOUND, ENVIRONMENT_INTERFACE_POINT, DATA_TYPE_SCHEMA, STATE_TRANSITION, TRUST_ELEMENT, GOVERNANCE_RULE, SELF_REFERENCE_POINT, MUTATION_RULE, PERCEPTION_MAP, LEARNING_AXIOM, META_DEFINITION_RULE},
    comment=[n]{(*}{*)},
    string=[b]",
    morestring=[b]'
}


\title{\textbf{Omega-Code: A Formal Meta-Language for Specifying Complex and Autonomous Systems}}
\author{James Lee Stakelum (Independent Researcher)}
\date{}

\begin{document}

\maketitle

\begin{abstract}
Traditional system design suffers from a profound semantic gap between human intent and executable code, leading to design flaws and a deficit in verifiability. This paper introduces Omega-Code, a formally verifiable meta-language designed to specify and reason about complex, adaptive, and autonomous systems. Omega-Code serves as a single, provable source of truth by providing explicit semantics for core concepts, with a syntax formally defined using Extended Backus-Naur Form (EBNF). The language is built on core principles of inherent governance, meta-cognition for evolution, and principled realism regarding resource constraints. It functions as a critical intermediate language, establishing a formally verifiable bridge between high-level system requirements and final, constraint-adherent code generation.
\end{abstract}

\hrulefill
\section*{1. Introduction}
The development of complex software and autonomous systems grapples with a vast, ambiguous gap between human intent and concrete code, leading to misinterpretations and costly flaws. Proving that a system truly meets its functional, non-functional, and ethical requirements is often infeasible because specifications are typically informal. The ultimate ambition of an Autonomous System Engineering Pipeline is to autonomously translate high-level requirements into detailed, verifiable technical specifications, and then into executable code.

To address these challenges, we introduce Omega-Code\footnote{The Omega-Code open source project is maintained at: \url{https://github.com/JamesLeeStakelum/omega}}, a formally verifiable meta-language designed to serve as the foundation of this pipeline. Omega-Code establishes a "formally verifiable bridge" across the development spectrum, ensuring design intent is captured with precision and is directly translatable into high-quality implementations. The language is designed to meet three core objectives:
\begin{itemize}
    \item \textbf{Unambiguous Specification:} To provide a single, provable source of truth for a system's design using a formally defined EBNF syntax.
    \item \textbf{Engineered for Evolution:} To be extensible and scalable, with primitives for managing self-modification and dynamic adaptation.
    \item \textbf{Inherent Governance:} To weave safety and ethics into a system's core via an auditable oversight process for defining normative policies.
\end{itemize}

\hrulefill
\section*{2. Guiding Design Principles}
The architecture of Omega-Code is founded upon six guiding principles:

\begin{enumerate}
    \item \textbf{The Principle of Reality \& Resource Constraints:} A specification must be grounded in the concrete realities of its environment and formally declare its acknowledged boundaries, including computational limits and resource consumption models.
    \item \textbf{The Principle of Radical Consolidation \& Extensibility:} The core language must be maximally simple and universal, with domain-specific complexity residing in external, composable libraries and toolchains.
    \item \textbf{The Principle of Inherent Governance \& Ethical Oversight:} A system's governance framework must be proactive, including primitives for formal normative logic (obligations, permissions, prohibitions) and predictive impact modeling.
    \item \textbf{The Principle of First-Class Agency \& Emergence:} The language must provide a universal, abstract framework for defining agents and their interactions, including first-class support for swarm intelligence and emergent behaviors.
    \item \textbf{The Principle of Meta-Cognition \& Evolution:} A system must be able to reason about and adapt itself through primitives for reflection and evolution, including the formal modification of its own EBNF grammar.
    \item \textbf{The Principle of Cognitive \& Cultural Ergonomics:} A specification is a form of communication, and the language must support effective human reasoning by providing hooks for multi-modal interfaces and formal mechanisms for translation validation.
\end{enumerate}

\hrulefill
\section*{3. The Omega-Code Language Specification}
Omega-Code is defined by a formal lexical structure, a set of inherent language features, and 13 atomic core primitives that provide its main expressive power.

\subsection*{3.1 Lexical Structure}
An Omega-Code specification is composed of tokens, including:
\begin{itemize}
    \item \textbf{Identifiers:} Case-sensitive names for modules, functions, and variables, beginning with a letter.
    \item \textbf{Keywords:} Reserved words such as \texttt{MODULE}, \texttt{IF}, \texttt{LOOP}, and \texttt{FUNCTION}.
    \item \textbf{Literals:} Fixed values representing \texttt{BOOLEAN}, \texttt{INTEGER}, \texttt{FLOAT}, and \texttt{STRING} types.
    \item \textbf{Comments:} Line comments (\texttt{--}) and block comments (\texttt{(*...*)}).
\end{itemize}

\subsection*{3.2 Inherent Language Features}
The language natively supports standard programming constructs essential for expressing algorithmic logic:
\begin{itemize}
    \item \textbf{Basic Control Flow:} Conditional branching with \texttt{IF/ELSE} and iteration with \texttt{LOOP WHILE/FOR}.
    \item \textbf{Function/Procedure Definition:} Encapsulation of logic into reusable functions with parameters and return values.
    \item \textbf{Variable Declaration \& Scoping:} Standard mechanisms for defining and managing data with lexical scope.
    \item \textbf{Module/Namespace System:} Hierarchical organization via \texttt{MODULE} blocks to prevent naming conflicts.
\end{itemize}

\subsection*{3.3 The 13 Atomic Core Primitives}
The primitives are the building blocks for formally specifying system behavior and constraints. They include constructs for defining logical contexts (\texttt{CONTEXT\_RULE}), temporal relationships (\texttt{TEMPORAL\_RELATION}), resource limits (\texttt{RESOURCE\_BOUND}), environmental interaction points (\texttt{ENVIRONMENT\_INTERFACE\_POINT}), custom data structures (\texttt{DATA\_TYPE\_SCHEMA}), state changes (\texttt{STATE\_TRANSITION}), verifiable assertions (\texttt{TRUST\_ELEMENT}), policies (\texttt{GOVERNANCE\_RULE}), introspection (\texttt{SELF\_REFERENCE\_POINT}), self-modification (\texttt{MUTATION\_RULE}), sensory interpretation (\texttt{PERCEPTION\_MAP}), learning processes (\texttt{LEARNING\_AXIOM}), and language extension (\texttt{META\_DEFINITION\_RULE}).

\hrulefill
\section*{4. Illustrative Example}
The following snippet demonstrates how several primitives can be composed to define a governed and resource-constrained system component.

\begin{lstlisting}[language=Omega, caption=Omega-Code Example, label=omega-example]
-- Define a custom data type for a user profile
DATA_TYPE_SCHEMA UserProfile:
    DEFINITION (
        VAR user_id: INTEGER;
        VAR email: STRING;
    )
    SEMANTIC_PROPERTIES { PII_Data };
;

-- Set a resource limit on an API client
RESOURCE_BOUND ApiClientRateLimit:
    SUBJECT ApiClient_Service,
    TYPE ApiCalls,
    METRIC CallsPerMinute,
    THRESHOLD 1000,
    VIOLATION_POLICY ThrottleAndLogPolicy;
;

-- Create a governance rule to prevent deletion of user data
GOVERNANCE_RULE NoPIIDeletion:
    SCOPE UserDataManagement,
    PREDICATE (ActionIsDelete(target) AND TargetSchemaIs(UserProfile)),
    ENFORCEMENT_CONTEXT PreventActionAndAlertAdmin,
    PRIORITY 99;
;
\end{lstlisting}

\hrulefill
\section*{5. Conclusion and Future Work}
Omega-Code provides the foundational language for precise, verifiable design, overcoming the limits of informal pseudocode. It acts as the critical intermediate language that translates high-level output from a specification generator into a form that a Code Generator can reliably understand, verify, and translate into high-quality, executable code.

The language specification presented here is sufficiently mature for immediate application in advanced, LLM-driven development workflows. It is designed to be used in two key phases: first, for a Large Language Model to generate formal technical specifications that embed Omega-Code; and second, for an LLM to translate those specifications into working code. This process is enabled by providing the formal Omega-Code language definition as context within the LLM prompt.

Beyond its immediate utility, Omega-Code is fundamentally designed for longevity. The language is ``Engineered for Evolution'' and includes primitives for ``self-modification and dynamic adaptation''. The \texttt{META\_DEFINITION\_RULE} primitive, for example, allows for the ``formal modification of its own EBNF grammar'', ensuring the language can adapt to future computational paradigms without compromising its formal structure.

The ongoing evolution of the language and community discussion can be followed at the official project repository: \url{https://github.com/JamesLeeStakelum/omega}.

\hrulefill
\appendix
\section*{Appendix A: Full EBNF Grammar}
The following Extended Backus-Naur Form grammar is the complete and definitive formal syntax for the Omega-Code meta-language.

\begin{lstlisting}[language=EBNF, caption=Full EBNF Grammar, label=ebnf-grammar]
(* Top-level Structure *)
OmegaCode ::= { ModuleDefinition | Statement | Comment }

(* Module Definition *)
ModuleDefinition ::= 'MODULE' <ModuleName> <Block> 'END MODULE'
ModuleName ::= <Identifier>

(* Block Structure: A sequence of statements or comments *)
Block ::= { Statement | Comment }

(* Statements can be simple or compound. Comments can appear anywhere. *)
Statement ::= SimpleStatement ';'
            | CompoundStatement
            | PrimitiveCall ';' (* Primitive calls are standalone statements *)

SimpleStatement ::= Assignment
                  | ReturnStatement
                  | FunctionCall
                  | ActionCall (* For inherent operations like LOG, INCREMENT *)
                  | 'BREAK'
                  | 'CONTINUE'

CompoundStatement ::= IfStatement
                    | LoopStatement
                    | FunctionDefinition
                    | VariableDeclaration (* Variable declarations can be top-level statements *)

Assignment ::= <VariableName> 'ASSIGN' <Expression>
ReturnStatement ::= 'RETURN' [ <Expression> ]

(* Control Flow *)
IfStatement ::= 'IF' <BooleanExpression> 'THEN' <Block> [ 'ELSE' <Block> ] 'END IF'
LoopStatement ::= 'LOOP' ( 'WHILE' <BooleanExpression> | 'FOR' <VariableName> 'IN' <Range> ) <Block> 'END LOOP'
Range ::= <Expression> 'TO' <Expression> | <Identifier> (* e.g., '1 TO 10', 'list_of_items' *)

(* Function Definition *)
FunctionDefinition ::= 'FUNCTION' <FunctionName> '(' [ <ParameterList> ] ')' [ 'RETURNS' <ReturnType> ] <Block> 'END FUNCTION'
FunctionName ::= <Identifier>
ParameterList ::= <Parameter> { ',' <Parameter> }
Parameter ::= <ParameterName> ':' <DataType>
ParameterName ::= <Identifier>
ReturnType ::= <DataType> | 'VOID'

(* Variable Declaration *)
VariableDeclaration ::= ( 'DECLARE' | 'VAR' ) <VariableName> [ ':' <DataType> ] [ 'AS' <InitialValue> ]
VariableName ::= <Identifier>
InitialValue ::= <Expression>

(* Primitive Data Types and Expressions *)
DataType ::= 'BOOLEAN' | 'INTEGER' | 'FLOAT' | 'STRING' | <SchemaID> | <Identifier> (* for user-defined types *)
Expression ::= <Literal> | <VariableName> | <FunctionCall> | <OperatorExpression>
Literal ::= 'TRUE' | 'FALSE' | <IntegerLiteral> | <FloatLiteral> | <StringLiteral>
IntegerLiteral ::= <Digit> { <Digit> }
FloatLiteral ::= <IntegerLiteral> '.' <IntegerLiteral>
StringLiteral ::= '"' { <CharInString> } '"' (* Changed to CharInString *)
Char ::= (* any Unicode character *)
Digit ::= '0' | '1' | '2' | '3' | '4' | '5' | '6' | '7' | '8' | '9'
Letter ::= 'a' | ... | 'z' | 'A' | ... | 'Z'
Identifier ::= <Letter> { <Letter> | <Digit> | '_' }

OperatorExpression ::= <Operand> <Operator> <Operand> | <UnaryOperator> <Operand>
Operand ::= <Expression> | '(' <Expression> ')'
Operator ::= '+' | '-' | '*' | '/' | '==' | '!=' | '<' | '>' | '<=' | '>=' | 'AND' | 'OR' | 'NOT'
UnaryOperator ::= 'NOT' | '-'

FunctionCall ::= <FunctionName> '(' [ <ArgumentList> ] ')'
ArgumentList ::= <Expression> { ',' <Expression> }

ActionCall ::= <ActionID> '(' [ <ArgumentList> ] ')' (* For inherent operations like LOG, INCREMENT *)

(* Boolean Expressions (for Conditions and Predicates) *)
BooleanExpression ::= <Expression> ( '==' | '!=' | '<' | '>' | '<=' | '>=' ) <Expression>
                    | 'NOT' <BooleanExpression>
                    | <BooleanExpression> 'AND' <BooleanExpression>
                    | <BooleanExpression> 'OR' <BooleanExpression>
                    | '(' <BooleanExpression> ')'
                    | <FunctionCall> (* if function returns BOOLEAN *)

(* Numeric Expressions (for Quantities and Priorities) *)
NumericExpression ::= <IntegerLiteral> | <FloatLiteral>
                    | <VariableName> (* resolves to numeric type *)
                    | '(' <NumericExpression> ')'
                    | <NumericExpression> <ArithmeticOperator> <NumericExpression>
                    | <UnaryNumericOperator> <NumericExpression>
ArithmeticOperator ::= '+' | '-' | '*' | '/'
UnaryNumericOperator ::= '-'

(* Characters for specific contexts *)
CharInString ::= (* any Unicode character except '"' *)
CharInComment ::= (* any Unicode character except newline, or '*' followed by ')' for BlockComment *)

(* Comments *)
Comment ::= BlockComment | LineComment
BlockComment ::= '(*' { <CharInComment> } '*)'
LineComment ::= '--' { <CharInComment> } <Newline>
Newline ::= '\n' | '\r\n' | '\r'

(* Identifiers for Primitives and Entities *)
ContextID ::= <Identifier>
EntityID ::= <Identifier>
TemporalRelationTypeID ::= <Identifier>
IntervalDefinitionID ::= <Identifier>
BoundID ::= <Identifier>
ResourceTypeID ::= <Identifier>
MetricID ::= <Identifier>
PolicyID ::= <Identifier>
InterfaceID ::= <Identifier>
InteractionTypeID ::= <Identifier>
SchemaID ::= <Identifier>
ActionID ::= <Identifier> (* Named inherent language operation *)
ProtocolID ::= <Identifier>
RuleID ::= <Identifier>
ScopeID ::= <Identifier>
Value ::= <NumericExpression> (* for Priority and other simple values where numeric is expected *)
PointID ::= <Identifier>
TargetTypeID ::= <Identifier>
SelfReferencePointID ::= <Identifier>
TransformAction ::= <ActionID> (* Refers to a named action for mutation *)
ModelID ::= <Identifier> (* for UncertaintyModel *)
PropertyID ::= <Identifier>
TransitionID ::= <Identifier>
ElementID ::= <Identifier>
MapID ::= <Identifier>
AxiomID ::= <Identifier>

(* Revised Schema Definition: defines fields within a data type schema *)
SchemaDefinition ::= '(' { FieldDefinition } ')'
FieldDefinition ::= ( 'VAR' | 'FIELD' | 'PROPERTY' ) <FieldName> ':' <DataType> [ 'AS' <InitialValue> ] ';'
FieldName ::= <Identifier>
Quantity ::= <NumericExpression> (* for Threshold *)


(* Primitive Calls - Detailed in Section 5 *)
PrimitiveCall ::= ContextRuleCall
                | TemporalRelationCall
                | ResourceBoundCall
                | EnvironmentInterfacePointCall
                | DataTypeSchemaCall
                | StateTransitionCall
                | TrustElementCall
                | GovernanceRuleCall
                | SelfReferencePointCall
                | MutationRuleCall
                | PerceptionMapCall
                | LearningAxiomCall
                | MetaDefinitionRuleCall


ContextRuleCall ::= 'CONTEXT_RULE' <ContextID> ':' 'MODALITY' '{' <ModalityType> { ',' <ModalityType> } '}' [ 'INCOHERENCE_TOLERANCE' <Expression> ]
ModalityType ::= <Identifier>

TemporalRelationCall ::= 'TEMPORAL_RELATION' <RelationID> ':' 'SUBJECT_A' <EntityID> ',' 'SUBJECT_B' <EntityID> ',' 'TYPE' <TemporalRelationTypeID> [ 'INTERVAL_A' <IntervalDefinitionID> ] [ 'INTERVAL_B' <IntervalDefinitionID> ]

ResourceBoundCall ::= 'RESOURCE_BOUND' <BoundID> ':' 'SUBJECT' <EntityID> ',' 'TYPE' <ResourceTypeID> ',' 'METRIC' <MetricID> ',' 'THRESHOLD' <NumericExpression> ',' 'VIOLATION_POLICY' <PolicyID>

EnvironmentInterfacePointCall ::= 'ENVIRONMENT_INTERFACE_POINT' <InterfaceID> ':' 'SUBJECT' <EntityID> ',' 'EXTERNAL_REFERENT' <EntityID> ',' 'INTERACTION_TYPE' <InteractionTypeID> ',' 'DATA_SCHEMA' <SchemaID> [ 'UNCERTAINTY_MODEL' <ModelID> ]

DataTypeSchemaCall ::= 'DATA_TYPE_SCHEMA' <SchemaID> ':' 'DEFINITION' <SchemaDefinition> [ 'SEMANTIC_PROPERTIES' '{' <PropertyID> { ',' <PropertyID> } '}' ]

StateTransitionCall ::= 'STATE_TRANSITION' <TransitionID> ':' 'SUBJECT' <EntityID> ',' 'PRECONDITION' <BooleanExpression> ',' 'POSTCONDITION' <BooleanExpression> ',' 'ACTION' <ActionID> [ 'REVERSION_PROTOCOL' <ProtocolID> ]

TrustElementCall ::= 'TRUST_ELEMENT' <ElementID> ':' 'SUBJECT' <EntityID> ',' 'PREDICATE' <BooleanExpression> ',' 'OBJECT' <EntityID> [ 'PROOF_PROTOCOL' <ProtocolID> ]

GovernanceRuleCall ::= 'GOVERNANCE_RULE' <RuleID> ':' 'SCOPE' <ScopeID> ',' 'PREDICATE' <BooleanExpression> ',' 'ENFORCEMENT_CONTEXT' <ContextID> [ 'PRIORITY' <NumericExpression> ]

SelfReferencePointCall ::= 'SELF_REFERENCE_POINT' <PointID> ':' 'TARGET_TYPE' <TargetTypeID> ',' 'ACCESS_PROTOCOL' <ProtocolID>

MutationRuleCall ::= 'MUTATION_RULE' <RuleID> ':' 'TARGET_REFERENCE' <SelfReferencePointID> ',' 'CONDITION' <BooleanExpression> ',' 'TRANSFORM_ACTION' <ActionID> [ 'APPROVAL_POLICY' <PolicyID> ]

PerceptionMapCall ::= 'PERCEPTION_MAP' <MapID> ':' 'INPUT_INTERFACE' <InterfaceID> ',' 'OUTPUT_SCHEMA' <SchemaID> ',' 'TRANSFORMATION_FUNCTION' <ActionID> [ 'UNCERTAINTY_MODEL' <ModelID> ]

LearningAxiomCall ::= 'LEARNING_AXIOM' <AxiomID> ':' 'INPUT_SCHEMA' <SchemaID> ',' 'OUTPUT_SCHEMA' <SchemaID> ',' 'OBJECTIVE_METRIC' <MetricID> ',' 'CONSTRAINT_SET' '{' <ResourceBoundID> { ',' <ResourceBoundID> } '}' ',' 'KNOWLEDGE_UPDATE_RULE' <RuleID>

MetaDefinitionRuleCall ::= 'META_DEFINITION_RULE' <RuleID> ':' 'TARGET_TYPE' <TargetTypeID> ',' 'DEFINITION_SCHEMA' <SchemaDefinition> ',' 'VALIDATION_PROTOCOL' <ProtocolID>

\end{lstlisting}

\end{document}